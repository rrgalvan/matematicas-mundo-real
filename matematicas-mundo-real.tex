
\documentclass[
  unknownkeysallowed %  ignore keyval errors (produced by Lato font)
]{beamer}

\usepackage[T1]{fontenc}
\usepackage[utf8]{inputenc}
\usepackage[spanish]{babel}

\usepackage{pgf,pgfpages}
\usepackage{tikz}
\usetikzlibrary{arrows,shapes,backgrounds,calc}
% Daniel's proposal for "uncovering" parts of a tikz-tree %
\tikzset{
  minimized/.style={scale=0.3},
  noshadowed/.style={opacity=1.0},
  shadowed/.style={opacity=0.4},
  invisible/.style={opacity=0},
  minimize on/.style={alt=#1{minimized}{}},
  visible on/.style={alt=#1{}{invisible}},
  shadow on/.style={alt=#1{shadowed}{noshadowed}},
  alt/.code args={<#1>#2#3}{%
    \alt<#1>{\pgfkeysalso{#2}}{\pgfkeysalso{#3}} % \pgfkeysalso doesn't change the path
  },
}

\usepackage{graphicx}
\usepackage{colortbl}
\usepackage{media9}

%% Beamer style.....
\mode<presentation>
{
  \usetheme{PHD}
  \setbeamercovered{transparent}
  \setbeamertemplate{items}[square]
}

%\usefonttheme[onlymath]{serif}

\beamertemplatenavigationsymbolsempty

\defbeamertemplate{enumerate item}{mycircle}
{
  %\usebeamerfont*{item projected}%
  \usebeamercolor[bg]{item projected}%
  \begin{pgfpicture}{0ex}{0ex}{1.5ex}{0ex}
    \pgfcircle[fill]{\pgfpoint{-0.1pt}{.65ex}}{1.1ex}
    \pgfbox[center,base]{\color{PHDyellow}{\insertenumlabel}}
  \end{pgfpicture}%
}
[action]
{\setbeamerfont{item projected}{size=\scriptsize}}
\setbeamertemplate{enumerate item}[mycircle]

%..............beamer style

\newcommand{\strip}[1]{%
  \begin{flushright}
    \color{PHDgrayC}
    \scriptsize{#1}
\end{flushright}}

\title[C\'adiz Num\'erica 2013]
{Las matemáticas y eso llamado <<mundo real>>}
\author[J. R. Rguez. Galv\'an]{%
  { J. Rafael Rodr\'{\i}guez Galv\'an}
  \\[1.5em]
  {\small \em Colegio Argantonio}
  \\[0.2em]
  {\scriptsize Cádiz, \today}
}
\date{}

% XeLaTeX font choosing
% \usepackage{fontspec}%{xltxtra} %fontspec}
% \setsansfont{Fontin Sans}
% \setsansfont{Lato}

% PDFLaTeX font choosing
\usepackage[math, default, scale=1.0]{lato}
% \usepackage[math, default, scale=0.95]{vollkorn}

% Different math fonts, see http://tug.org/pracjourn/2006-1/hartke/hartke.pdf
%\usepackage{eulervm}
%\usepackage{ccfonts, eulervm}
%\usepackage[math]{kurier}
%\usepackage[math]{anttor}
%\usepackage{pxfonts}
%\usepackage{mathpazo}
%\usepackage{mathpple}
%\usepackage[varg]{txfonts}
%\usepackage{arev}
%\usepackage{fourier}

\usepackage{tabularx}
\usepackage{array, multirow, booktabs, rotating} % booktabs: toprule, midrule...



%\newtheorem{remark}{Remark}
%\newtheorem{theorem}{Theorem}

\newcommand\gris{\color{PHDgray}}
\newcommand\amarillo{\color{PHDyellow}}

\setcounter{tocdepth}{1}


%======================================================================
\begin{document}
%======================================================================

%
% Portada ............................................................
%
\setbeamertemplate{background}
 {\includegraphics[width=\paperwidth,height=\paperheight]{frontpage_bg}}
\setbeamertemplate{footline}[default]

% Write custom titlepage ------->>>
\begin{frame}
  \titlepage
  \vspace{2.5cm}
\end{frame}

%
% Contenidos ............................................................
%

% Set the background for the rest of the slides.
\setbeamertemplate{background}
 {\includegraphics[width=\paperwidth,height=\paperheight]{slide_bg}}

\begin{frame}{Plan}
  \tableofcontents
\end{frame}

\section{Introducción}

\begin{frame}{El <<mundo de las ideas>> y <<mundo real>>}
  \begin{minipage}{0.45\linewidth}
    \vspace{-1.5em}
    % \begin{tikzpicture}
    %   \draw
    %   (3,-1) coordinate (a) node[right] {a}
    %   -- (0,0) coordinate (b) node[left] {b}
    %   -- (2,2) coordinate (c) node[above right] {c}
    %   pic["$\alpha$", draw=orange, <->, angle eccentricity=1.2, angle radius=1cm]
    %   {angle=a--b--c};
    % \end{tikzpicture}
    $$
    (a-b)(a+b)=a^2-b^2 \qquad
    $$
    \begin{center}
      \includegraphics[width=0.65\linewidth,height=0.6\linewidth]{img/sin-cos}
    \end{center}
    \vspace{-1em}
    \begin{align*}
    \int_a^b f(x) \; dx = F(b)-F(a)
      \\[2em]
      \small
      \left\{
      \begin{aligned}
        \frac{\partial \mathbf{u}}{\partial t} - \nu \Delta \mathbf{u}
        + \nabla p &= \mathbf{f} \quad \text{en } \Omega\subset\mathbb{R}^n,
        \\
        \nabla\cdot \mathbf{u} &= 0 \quad \text{en } \Omega\subset\mathbb{R}^n.
      \end{aligned}
                                 \right.
    \end{align*}
  \end{minipage}
  \quad
  \begin{minipage}{0.45\linewidth}
    \includegraphics[width=0.8\linewidth]{img/setas}
    \\[-1.5em]
    \hspace*{2em}%
    \includegraphics[width=0.8\linewidth]{img/world-trade-center}
    \\[-1em]
    \hspace*{3.5em}%
    \includegraphics[width=0.8\linewidth]{img/ciclon}
  \end{minipage}
  \begin{minipage}{0.45\linewidth}
  \end{minipage}
\end{frame}

\begin{frame}{Internet...}
    \begin{center}
      \begin{tikzpicture}
        \node[shadow on=<2->] {\includegraphics[width=0.83\linewidth]{img/internet}};
        \node<2-> at (0,1.5) {\textbf{3.625.728.487} usuarios en el
          mundo\footnote{\color{gray}Datos recogidos el 3/5/2017, de
            \url{http://www.internetlivestats.com}}};
        \node<2-> at (0,0.6) {\textbf{1.186.955.219} sitios web\footnotemark[\value{footnote}]};
        \node<3> at (0, -0.6) {{\large\emph{...¿cómo encuentras la}}};
        \node<3> at (0, -1.4) {{\LARGE\emph{\textbf{\alert{información que te interesa}}}}};
        \node<3> at (0, -2.3) {{\Huge\emph{\textbf{?}}}};
      \end{tikzpicture}
    \end{center}
\end{frame}

\begin{frame}{Buscadores web}
  \vspace*{-1em}
  \begin{flushright}
    \begin{tikzpicture}
      \node[minimize on=<2-3>] {\includegraphics[width=0.9\linewidth]{img/google}};
    \end{tikzpicture}
  \end{flushright}
  \vspace*{-4.9em}
  \begin{flushleft}
    \begin{tikzpicture}
      \node<2->[minimize on=<4>] {\includegraphics[width=0.6\linewidth]{img/page-rank}};
      \node<2-3> at (5.5,-1.9) {\large \it {\bfseries\href{https://es.wikipedia.org/wiki/Algoritmo}{\structure{Algoritmo}}} <<PageRank>>};
      \node<2-3> at (4.05,-3.0) {\small $PR(A) = {1 - d \over N} + d \left( \frac{PR(B)}{L(B)}+ \frac{PR(C)}{L(C)}+ \frac{PR(D)}{L(D)}+\,\cdots \right)$};
      \node<3> at (2.1,-4.1) {\textbf{Enorme \textit{sistema de ecuaciones}}: ¡una incógnita por cada sitio web!};
      \node<3> at (3.5,-4.6) {\small\color{gray}...aunque existen métodos eficientes para resolverlos};
      \node<4> at (5.2,-1.3) {
        \begin{tabular}{c}
          \large
          !El \textit{éxito de los buscadores web} se basa en las
          \\[1.2em]
          \Huge \textbf{matemáticas}!
        \end{tabular}
      };
    \end{tikzpicture}
  \end{flushleft}
\end{frame}

\begin{frame}
  \Large
  \begin{center}
    ~
    \\[2em]
    En la vida diaria
    \\[1.5em]
    hay muchas más matemáticas
    \\[1.5em]
    de lo que se podría pensar.
    \\[4em]
    \small
  \onslide<2>{\href{http://www.cedya2017.org/sesiones_especiales.html}{Sesiones especiales en el congreso <<XXV CEDYA / XV CMA>>}}
  \end{center}
\end{frame}

\begin{frame}{Modelos atmosféricos}
  \begin{minipage}{0.47\linewidth}
    \begin{flushright}
      \onslide<1-3>{\color{darkgray}\small\emph{¿Cómo se comporta ese fluido al que llaman
          <<atmósfera>>?}}
    \end{flushright}
    \vspace{3em}
    \onslide<2->{
      \begin{equation*}
        \left\{
          \begin{aligned}
            \frac{\partial \alert<4>{\mathbf{u}}}{\partial t}
            + (\mathbf{\alert<4>u}\cdot\nabla)\mathbf{\alert<4>u}
            - \nu \Delta \mathbf{\alert<4>u}
            + \nabla \alert<3>p &= \mathbf{f},
            \\
            \nabla\cdot \mathbf{\alert<4>u} &= 0.
          \end{aligned}
        \right.
      \end{equation*}
      \begin{flushright}
        {\scriptsize\color{darkgray}Modelo simplificado (ecuaciones de
          Navier-Stokes)}
      \end{flushright}
    }
    \vspace{1.1em}
    \begin{flushright}
     \onslide<3>{\alert<3>{P}resión atmosférica}
     \\[0.5em]
     \onslide<4>{\alert<4>{V}elocidad del fluido}
      \end{flushright}
  \end{minipage}
  \hfill
  \begin{minipage}{0.48\linewidth}
    \includegraphics<1-3>[width=1.0\linewidth,height=1.3\linewidth]{img/tiempo-atmosf}
    \includegraphics<4>[width=1.0\linewidth,height=1.2\linewidth]{img/modelos-tiempo-atmosf}
  \end{minipage}
\end{frame}

\begin{frame}{Flujo de aguas en el Estrecho de Gibraltar}
  \begin{minipage}{0.45\linewidth}
    \begin{itemize}
    \item Aproximación del dominio de cálculo (triangulación)
    \item Aproximación numérica de de las ecuaciones (Navier-Stokes)
    \item Resolución de las ecuaciones en los vértices de los
      triángulos
    \end{itemize}
  \end{minipage}
  \hfill
  \begin{minipage}{0.5\linewidth}
    \includegraphics<1->[width=1.0\linewidth,height=0.65\linewidth]{img/malla-estrecho-2d}
  \end{minipage}
  \vspace*{2em} \onslide<2>{
    Cientos de miles de incógnitas {\color{gray}\dotfill} \\[1em]
    {\color{gray}\dotfill} no linealidad {\color{gray}\dotfill} un mundo 3D {\color{gray}\dotfill}\\[1em]
    {\color{gray}\dotfill} \large
    ¡\textit{\bfseries\structure<2>{grandes~ordenadores},~\structure<2>{supercomputación}}!}
\end{frame}

\begin{frame}
  \includemedia{alt content}{video/gibraltar-tubes-3d-30s.mp4}
\end{frame}
\begin{frame}{Modelos matemáticos}
  \begin{center}
      \begin{tikzpicture}[
        % Define standard arrow tip
        >=stealth',
        % Define style for boxes
        punkt/.style={
           rectangle,
           rounded corners,
           scale=1.1, fill=gray!30,
           draw=black, very thick,
           text width=6.5em,
           minimum height=2em,
           text centered},
         % Define arrow style
         pil/.style={
           ->,
           thick,
           shorten <=2pt,
           shorten >=2pt},
         % line width=1.2pt
        ]
        \node[punkt] (A) at (0,0) {\color{PHDgreen}<<Mundo real>>};
        \node[punkt] (B) at (6,0) {\color{PHDred}Modelo matemático}
        edge[pil, <-, bend left=-45] (A);
        \node[punkt] (C) at (3,-3) {\color{orange}Predicciones}
        edge[pil, ->, bend left=45] (A)
        edge[pil, <-, bend left=-45, inner sep=1.0ex, align=right] node[right] {hola} (B);
        % \path[line width=1.2pt, ->] (A) edge (B);
        % \path[line width=1.2pt, ->] (B) edge (C);
        % \path[line width=1.2pt, ->] (C) edge (A);
      \end{tikzpicture}
  \end{center}
\end{frame}

\begin{frame}{Imágenes utilizadas}
  \scriptsize
  \begin{enumerate}
  \item Tropical Cyclone Debbie Make Landfall in Queensland. NASA, licencia CC-by.
  \item King Boletes (Boletus edulis). Bernard Sprag, dominio público.
  \item One World Trade Center\_2016. Harvey Barrison, licencia CC-by-sa.
  \item Sine and Cosine. MediaCommons, User:345Kai, dominio público.
  \item Internet. Procsilas Moscas, licencia CC-by.
  \item Google.Paul Downey, licencia CC-by.
  \item Page Rank. Felipe Micaroni Lalli, licencia CC-by-sa.
  \item Madrid 264 weather girl. David Holt, licencia CC-by-sa.
  \item Isobaras. MediaCommons, User:Asierog, licencia CC-by-sa.
  \item NOAA Wavewatch III 120-hour wind and wave forecast for the
    North Atlantic. Public domain.
  \end{enumerate}
\end{frame}

\end{document}

%%% Local Variables:
%%% coding: utf-8
%%% TeX-master: t
%%% mode: latex
%%% ispell-local-dictionary: "spanish"
%%% End:
